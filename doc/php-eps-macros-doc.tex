\documentclass[a4paper]{scrartcl}

\title{\texttt{php-eps-macros}}
\subtitle{A set of helper macros in PHP to write EPS}
\author{Romaric Pujol\\\footnotesize{\texttt{romaric.pujol@insa-lyon.fr}}}
\date{September 28, 2013}

\usepackage{listings}

\newcommand\code[1]{\lstinline{#1}}

\lstset{showspaces=false,showstringspaces=false}
\lstset{basicstyle=\footnotesize\ttfamily}

\begin{document}
\maketitle
\section{Introduction}
This document describes a set of PHP helper files and scripts to generate EPS
   (Encapsulated PostScript) files.
When writing documents in \LaTeX, the author might want to include figures in
   it.
Figures can be generated in many ways by many existing softwares, but the
   author might want to make its own from scratch.
If she likes programming, the PostScript language is a wonderful option to
   consider seriously.

In this document, we assume that the reader is already familiar with EPS files.

\section{Directory Content}
\begin{itemize}
\item \texttt{coords.php}
\item \texttt{eps-header.php}
\item \texttt{fatal.php}
\item \texttt{SmartInsertFont.php}
\end{itemize}

\section{Structure of the PHP file}
The PHP file should start as follows:
\lstset{language=PHP}
\begin{lstlisting}
<?php
   $lx=0;
   $ly=0;
   $ux=200;
   $uy=300;
   $author="Romaric Pujol";
   $title="An EPS example file";
   $date="2013/09/28";
   require_once("eps-header.php");
?>
\end{lstlisting}

Then, running \code{php file.php} will output the following EPS header:
\begin{lstlisting}
%!PS-Adobe-2.0 EPSF-3.0
%%BoundingBox: 0 0 200 300
%%Creator: Romaric Pujol
%%CreationDate: 2013/09/28
%%Title: An EPS example file
%%Pages: 1
%%EndComments
\end{lstlisting}

Afterwards, you're free to mix PHP and EPS.
\end{document}

